\documentclass{projdoc}
\organization{Avans University of Applied Sciences}
\project{Project cr\^epe}
\author{%
	Loek Le Blansch\and%
	Wouter Boerenkamps\and%
	Jaro Rutjes\and%
	Max Smits\and%
	Niels Stunnebrink%
}


\title{Research document}

\begin{document}
\tablestables
\newpage

\section{Introduction}

\section{Game engine}

\subsection{Introduction}

To build a game engine, it must first be understood how it operates. The
functionalities it requires and how these functionalities work together must be
determined. In this section, the general functioning of a game engine and the
different parts required are described.

\subsection{Findings}

A game engine is not the game itself but a platform with which games are built. It
should provide the functionalities with which the game is constructed. The purpose of
a game engine is not to create data out of nothing. Instead, data is read, and the
correlating features and effects are generated. However, the engine is also used to
create these files, referred to as ``assets.'' The game engine must be able to accept
a certain format of these assets---whether levels, sprites, or textures---and convert
them into usable data.

\subsubsection{Layers}

A game engine is composed of multiple layers, each with its own functions. These
layers are divided into the following categories:\noparbreak
\begin{description}
	\item[Resource manager] Responsible for what happens when the engine is launched,
		including loading data formats.
	\item[Application] Manages the run loop, time, code execution, and events
		(e.g.~input events).
	\item[Window/\glspl{hid}] Handles input and events.
	\item[Renderer] Responsible for drawing the necessary objects on the screen,
		usually once per frame.
	\item[Debugging support] Provides testing for the engine, such as logging or
		performance profiling.
	\item[Scripting layer] Runs scripts, such as Lua or Python.
	\item[Memory systems] Handles and monitors memory usage.
	\item[\gls{ecs}] Provides a modular way to create game objects, add physics, and
		define how the engine interacts with objects.
	\item[Physics] Adds specific physics to objects.
	\item[Audio] Processes audio.
	\item[AI] Provides artificial inteligent behavior.
\end{description}

\subsubsection{ECS}

A game engine must have the ability to keep track and update several game objects. To
do this most game engines employ an \gls{ecs} model which uses modulair components to
give entities properties and features. The need for an entity component system arises
because multiple game objects are required to create a scene in a game. These game
objects exist within the scene and perform actions, such as a UI display for a score.
This game object does not need to be rendered; it could be a script running in the
background. It could also be a player sprite that is controlled. These entities need
to be aware of other entities, for example, during collisions. For this to function,
a scene is required to host all game objects. Within this scene, the game objects
must be stored efficiently, and entities must be provided with the required behavior,
such as audio, position, or physics. To create diverse entities with specific
functions: A scene can contain many different kinds of entities, each with different
properties and functions. But no matter how different each entity is, it remains an
entity. To assign properties and functions to entities, components are used. Entt is
an example of an \gls{ecs}.
% TODO: ref?entt

\subsection{Conclusion}

\section{Third-party Tools}

\subsection{Introduction}

Developing a game engine from scratch requires a significant amount of time, as many
different features are necessary. Fortunately, some of these features have already
been developed and can be reused in the form of frameworks and third-party
tools/libraries. The decision to use third-party libraries, and the selection of
which ones to use, directly influences the development process of the game engine. In
this section, several third-party frameworks and tools available for use are
described.

\subsection{Findings}

\subsubsection{Media Frameworks}

A game engine must have the ability to handle user input, render graphics, and
process audio. Several large frameworks are available that provide these features and
are already widely used by other game engines. The two most popular and
best-supported options are \gls{sdl2} and \gls{sfml}.

\paragraph{SDL2}

% TODO: ref?sdl2
According to its official website, \gls{sdl2} is \emph{``a cross-platform development
library designed to provide low-level access to audio, keyboard, mouse, joystick, and
graphics hardware via \gls{opengl} and \gls{d3d}. It is used by video playback
software, emulators, and popular games, including Valve's award-winning catalog and
many Humble Bundle games.''} \gls{sdl2} is written in the C programming language, and
therefore, structs and functions are used instead of objects and methods.

The advantages of \gls{sdl2} are:\noparbreak
\begin{itemize}
	\item Controller support is provided.
	\item 2D and 3D rendering are supported.
	\item Broad multiplatform support is offered, including older consoles such as the
		Wii.
	\item Low-level control is available.
	\item A large community ensures wide usage.
	\item Extended libraries can be used to add functionalities, such as SDL\_Mixer for
		sound.
\end{itemize}

The disadvantages of \gls{sdl2} are:\noparbreak
\begin{itemize}
	\item A limited built-in 2D renderer is provided.
	\item Extended libraries require setup.
\end{itemize}

\paragraph{SFML}

\gls{sfml} is a simple framework consisting of five modules: audio, graphics,
network, system, and window. This framework, written in C++, was designed to simplify
game development.

The advantages of \gls{sfml} are:
\begin{itemize}
	\item Object-oriented design is provided since it is written in C++.
	\item A built-in 2D renderer is available for ease of use.
	\item A built-in audio system is included.
	\item Cross-platform support is available for Linux, Windows, and macOS.
	\item Networking capabilities are provided for multiplayer or networked
		applications.
\end{itemize}

The disadvantages of \gls{sfml} are:
\begin{itemize}
	\item The 2D rendering engine may experience performance issues in large-scale
		games.
	\item The community is smaller compared to \gls{sdl2}.
	\item No native 3D support is provided.
	\item Not all image formats are supported.
\end{itemize}

\subsubsection{Audio}

for audio some options could be: FMOD, Wwise, or iirKlang

\subsection{Conclusion}

\section{resource manager}

\subsection{Introduction}

\subsection{Findings}

\subsection{unity formats}
Unity has many different asset file types that can be imported to use for a game \href{https://docs.unity3d.com/Manual/BuiltInImporters.html}{unity imports}. The most important formats are the audio, text and sprite formats.

\paragraph{Audio}

Unity uses a *Audio Source* component that holds the reference of the audio format which are all these different types.
\begin{itemize}
	\item ogg.
	\item aif.
	\item aiff.
	\item flac.
	\item wav.
	\item mp3.
	\item mod.
	\item it.
	\item s3m.
	\item xm.
\end{itemize}

Where then the *Audio clip* contains all the audio data used by the *Audio Source* components. Inside the audio clip are options about each specified audio file to change specific settings from the origanl file. for example load type, compression format, quality, sample rate and more \href{https://docs.unity.cn/2018.4/Documentation/Manual/class-AudioClip.html}{audio clip}

\paragraph{Sprite formats}
Unity supports many different image formats:
\begin{itemize}
	\item jpg.
	\item jpeg.
	\item tif/tiff.
	\item tga.
	\item gif.
	\item png.
	\item psd.
	\item bmp.
	\item iff.
	\item pict.
	\item pic.
	\item pct.
	\item exr.
	\item hdr.
\end{itemize}


\subsection{Conclusion}

\section{Rendering}

\subsection{Introduction}

\subsection{Findings}

\subsection{Conclusion}

\section{Event manager}

\subsection{Introduction}

\subsection{Findings}

\subsection{Conclusion}

\section{Memory/debugging}

\subsection{Introduction}

\subsection{Findings}

\subsection{Conclusion}

\section{Physics/scripting}

\subsection{Introduction}

\subsection{Findings}

\subsection{Conclusion}

\section{Conclusion}

\section{Gameobjects/components}

\subsection{Introduction}

\subsection{Findings}

\subsection{Conclusion}

\section{Conclusion}

\end{document}
