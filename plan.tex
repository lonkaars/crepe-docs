\documentclass{projdoc}
\organization{Avans University of Applied Sciences}
\project{Project cr\^epe}
\author{%
	Loek Le Blansch\and%
	Wouter Boerenkamps\and%
	Jaro Rutjes\and%
	Max Smits\and%
	Niels Stunnebrink%
}


% @Jaro: Zie hieronder
% Version 0.0 seems not right
% I'm also missing a 'version table'
% Shall we use this version system?:
% Version numbers after the comma indicate an interim version and version numbers
% before the comma indicate versions ready for publication

\title{Project Plan}
\version{0.1}

\begin{document}
\tablestables
\newpage

\section{Background}

This project is part of the fourth-year minor `Systems Programming in C++'. The minor
consists of several courses troughout 20 weeks, which create the basis for this
project. The project is focussed on building a game engine using the programming
language C++. This document describes our plan of attack for the project.

\section{Problem Definition}

% The assignments will be discussed with the project supervisor on specifics for this
% project. This will give the team the ability to direct the project.

% @Jaro: Ik snap niet wat je met deze zinnen wil zeggen. Het lijkt een soort van
% background information, maar toch niet helemaal ofzo. Hoe dan ook heb ik het er nu
% even uit gehaald en een nieuw kopje background toegevoegd.

\subsection{Problem Analysis}

CodedFun Games is a small, single-person game company looking to scale up. The owner,
who is also a game programmer, and graphical artist, has received government funding,
which he wants to invest in a custom-built game engine. The owner has no interest in
developing or maintaining a game engine himself, so he has hired a part-time engine
programmer. This programmer does not have time to create an entire engine but is
willing to maintain or expand an existing engine one day per week.

The client seeks a custom game engine that is easy to maintain, easy to extend, and
user-friendly. Additionally, the engine should be well-documented, which is
considered an essential aspect of being user-friendly.

So far, the client has made all his games in Unity and is very fond of the structure.
Therefore, he wants the new engine to adhere to a similar structure. A simple
requirements document is defined to specify how strictly this structure should be
followed.

Finally, because the client does not want to dive deeply into the engine himself, he
wants a secondary application (preferably a game) that can be used to test the
engine's features. This is referred to as the \emph{validation application}.

\subsection{Goal}

The goal is to develop a custom game engine that meets the client's requirements for
maintainability, extensibility, user-friendliness, and adherence to a Unity-like
structure. In addition, a validation application should be created to show and test
the engine's features.

\subsection{Result}

The expected result is a well-documented, custom game engine that follows a structure
similar to Unity. Additionally, a validation application should be provided to test
and showcase the engine's capabilities.

\section{Planning}

The customer specified multiple deliverables troughout the 20 weeks of this project.
Despite the fact that 20 weeks have been reserved for this project, the intention is
that the game will be delivered in week 17. Any delay/resit will take place between
weeks 17 and 20. A rough planning can be found in \cref{tab:planning}.

\begin{table}
	\begin{tabularx}{\linewidth}{lX}
		\toprule
		\textbf{Week\#} & \textbf{Deliverable}\\
		\midrule
		4 & Project Plan\\
		7 & Requirements document\\
		10 & POCs and Design document\\
		17 & Game engine, Validation application, Research document, and Qualification
		document\\
		\bottomrule
	\end{tabularx}
	\caption{Planning}
	\label{tab:planning}
\end{table}

\section{Risks}

\subsection{Techincal Risks}

\begin{description}
	\item[Multiplatform] The team works in both linux and windows, which poses a risk
		for the development if there is a platform dependency.
	\item[Integration] Users can make a wrong integration causing for delay or risk of
		losing code.
\end{description}

\subsection{Project Management Risks}

\begin{description}
	\item[Scope Expansion] There is a risk of creating a scope that is bigger than the
		requirements. Lack of Team Collaboration: Insufficient collaboration among team
		members may hinder progress.
\end{description}

\subsection{Measures}

\begin{description}
	\item[Multiplatform] The team can switch any time to a single platform, so this
		risk is metigated. 
	\item[Integration] By following standards and having an integrator, which checks
		every pull request, this risk is minimilised.
	\item[Scope Expansion] By writing detailed requirements and having weekly team
		meetings, to check if the progress is within the scope, should be sufficient to
		decrease the risk.
	\item[Lack of Team Collaboration] Weekly team meetings will result in collaboration
		among team members and discussing what each other tasks is, will decease this
		risk.
\end{description}

\section{Documentation}

This section describes the required documentation for the project, including the
types of documents to be created and the standards they must adhere to.

\subsection{Documents}

This project consists of seven main documents:\noparbreak
\begin{description}
	\item[Project Plan] Contains all elements related to the organization of the
		project, including timelines, milestones, roles, and responsibilities.
	\item[Requirements] Details the requirements and user stories, including both
		functional and non-functional requirements.
	\item[Research] Consists of all research related to this project.
	\item[Design] Describes the design choices, including architecture, user interface,
		and system components.
	\item[Qualification] Includes test cases, test plans, and quality measures to
		ensure the project meets its requirements and standards.
	\item[Working hours] A table which includes all working hours of each team member.
	\item[API Reference] Details the available endpoints, request and response formats,
		authentication methods, error codes, and examples for interacting with the
		project's API.
\end{description}

\subsection{Documentation Standard}

The documentation standard can be found in the contributing.md
\autocite{crepe:docs-standard}.

\section{Work Agreements}

Work agreements are the expectations and commitments made by the team members. This
section includes details on roles and responsibilities, documentation of work hours,
protocols for handling absences or delays, guidelines for addressing inconsistent
participation, and procedures for weekly updates and meetings. All team members
reviewed and agreed to these terms.

\subsection{Project roles}

\begin{description}
	\item[Loek Le Blansch] Integrator
	\item[Wouter Boerenkamps] Project Member
	\item[Jaro Rutjes] Team Leader / Scrum Master
	\item[Max Smits] Project Member
	\item[Niels Stunnebrink] Project Member
\end{description}

\subsection{Work hours}

Each project member will keep track of their own working hours and add them to the
`file'.

\subsection{Absence or delay}

If a project member is going to be absent or delayed, they are required to 
notify the team through either WhatsApp or Outlook. Additionally, the teacher 
should be informed of the absence as well.

\subsection{Inconsistent participation}

Inconsistent participation will be addressed in a structured manner:

\begin{description}
	\item[Initial Discussion] The team leader will first discuss the issue of
		inconsistent participation with the individual team member.
	\item [Team Discussion] If no improvement is observed, the issue will be brought up
		with the entire team to seek a collective solution.
	\item [Project Supervisor Involvement] Should the problem persist, despite these
		efforts, it will be escalated to the project supervisor (Bob van der Putten) for
		further action.
\end{description}

Valid reasons for absence or inconsistency will be considered, and no repercussions
will be necessary in such cases. However, if the inconsistency is due to other
factors, potential repercussions may include additional assignments or actions as
determined by the project supervisor.

A team member is considered to have inconsistent participation if their hours 
are significantly behind the team’s average or if tasks are not completed 
without valid reasons. It is essential that any concerns regarding a team 
member's performance, should be resolved through unanimous agreement within the team.

\subsection{Weekly Update}

Each team member is required to send their technical weekly update to Jaro Rutjes
before 12:00 on Friday. Personal updates may also be included and will be
confidential between the team member, the Team Leader, and/or the teacher, if
requested.

Jaro Rutjes will compile and send a team update to the project supervisor, Bob 
van der Putten, also by Friday. This update will include technical information 
on what has been accomplished, plans for upcoming work, and an overview of the 
team's overall progress.

\subsection{Weekly Meetings}

The project team will hold at least two meetings each week, with each meeting lasting
a maximum of 30 minutes. Following these meetings, an additional one hour will be
scheduled for discussions on the topics covered.

Meetings will be planned and discussed one week in advance, with invitations sent via
Outlook. Additional meetings may be scheduled if necessary to address issues or
project needs.

% Information about how and when Scrum will be used in this project (using Miro).
\section{Scrum (Miro)}

The team will start using scrum after the initial phase, of structering the project,
is finished. This phase is over when proof of concepts are being made.

\subsection{Scrum Board}

The Scrum board \autocite{miro:scrum-board} will consist of the following tabs:

\begin{description}
	\item[Backlog] This tab contains a list of all tasks and user stories that are
		planned for future sprints.
	\item[Next Sprint] This tab includes tasks and user stories that have been selected
		for the upcoming sprint.
	\item[Current Sprint] This tab displays the tasks and user stories that are
		actively being worked on in the current sprint.
	\item[In Progress] Tasks that are actively being worked on are moved to this tab.
	\item[Review] Completed tasks that are awaiting review or testing will be placed in
		this tab.
	\item[Done] Once tasks have been reviewed and are considered complete, they are
		moved to the Done tab.
	\item [Blocked] This tab is used for tasks that cannot proceed due to obstacles or
		dependencies.
\end{description}

To manage tasks effectively:

\begin{itemize}
	\item A task from the \emph{Current Sprint} tab should be selected and moved to the
		\emph{In Progress} tab when work begins. 
	\item The status of the task should be updated to \emph{In Progress} as soon as
		work starts.
	\item Once the task is completed and reviewed, it should be moved to the
		\emph{Done} tab, and its status should be updated to \emph{Done}.
\end{itemize}

Each task or user story will be assigned user points, which indicate the relative
size or complexity of the task compared to these examples.
% TODO: add examples

\subsection{Burn Down Chart}

The Burn Down Chart will be generated using Excel from the Scrum board data every
week. Each user story or task marked as done, will burn the chart downwards. This
Burn Down Chart is shared in the weekly updates with the team and the project
supervisor.

% Information regarding the git workflow of this project.
% TODO: Diagram explaining the workflow
\section{Git Workflow}

GitHub is used for version management of both code \autocite{crepe:code-repo} and
documentation \autocite{crepe:docs-repo}, each in its own respective repository. This
keeps the documentation and code seperate, resulting in ordened and manageable
repositories.

\subsection{Git New Branch}

% TODO: add here details from contributing.md?

\subsection{Git Merge To Master}

% TODO: add here diagram on what actions are taken before merge.

\subsection{Code Standard}

The code standard can be found in the contributing.md \autocite{crepe:code-standard}.

\end{document}
