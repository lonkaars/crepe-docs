\documentclass{projdoc}

\title{Software Design}

\begin{document}
\tablestables
\newpage

\section{Introduction}

This document outlines the design and development process of the cr\^epe game engine,
detailing the key decisions made during its creation. The primary goal of this engine
is to offer a streamlined, Unity-like experience tailored for developing 2D games
similar to Jetpack Joyride.

The cr\^epe engine is designed to ease the transition for developers familiar with
Unity, ensuring minimal friction when switching platforms. Our aim is to preserve
many of Unity’s core features while introducing a lightweight and open-source
alternative, licensed under the MIT License.

The engine is primarily aimed at indie developers who have prior experience with
Unity and are looking for a flexible, cost-effective solution with familiar
workflows.

\section{Overview}

As described above, the cr\^epe game engine's goal is to offer a Unity-like
experience tailored for developing 2D games similar to Jetpack Joyride. That is why
Jetpack Joyride and Unity provided the main inputs for this game engine design.
Firstly, a quick overview will be given of the Unity game engine, in particular the
\gls{ecs}. Secondly, this Overview will quickly talk you through some of the most
important parts of the game engine, and why these parts are needed to create the
Jetpack Joyride game.

\subsection{ECS}

The Unity game engine is structured using the Entity Component System (\gls{ecs}) (as
shown in \cref{fig:ecs-block-diagram}). The \gls{ecs} is made out of three main
subsystems, namely entities, components and systems. Entities are just IDs. An entity
is also called a GameObject in Unity and it is made out of one (or more) components.
Components are the classes that hold the data. The components determine what kind of
entity it is (e.g. an enemy, audio, and so on). Systems take care of the behavior of
the entities. Systems mainly read and write the enity's components data. The
\gls{ecs} clearly distinguishes the data (components) from the functionality
(systems).

\begin{figure}
	\centering
	\includegraphics[width=0.5\textwidth]{img/ECSBlockDiagram.png}
	\caption{ECS design pattern}
	Source: \autocite{img:ecs-block-diag}
	\label{fig:ecs-block-diagram}
\end{figure}

The \gls{ecs} will also be used at the cr\^epe game engine. Everything (from the
protagonist and bullets to the walls and enemies) in the cr\^epe game engine will be
a GameObject (i.e.~entity). The game programmer must program his game by creating all
kind of GameObjects and placing them in one (or multiple) scenes, just like Unity.

\subsection{Jetpack Joyride}

Firstly, some background information about Jetpack Joyride. Jetpack Joyride is a
side-scrolling endless runner action video game created by Halfbrick Studios. The
protagonist is called Barry Steakfries, who the player controls as he steals a
bullet-powered jet pack from a top-secret laboratory
\autocite{wikipedia:jetpack-joyride}. A screenshot from the game can be seen in
\cref{fig:jetpack-joyride} (pleae be aware that the goal of this project is not to
create an exact replica of Jetpack Joyride, it is only used as a source of
inspiration).

\begin{figure}
	\centering
	\includegraphics[width=0.5\textwidth]{img/JetpackJoyride.jpg}
	\caption{Jetpack Joyride}
	Source: \autocite{img:jetpack-joyride}
	\label{fig:jetpack-joyride}
\end{figure}

The protagonist wears a jetpack with which he can float in the air. The player must
avoid obstacles (such as lasers, missiles and zappers) by floating at the right
height. The player can control the protagonist's jetpack, thereby also controlling
the protagonist's height. The protagonist experiences gravity and other forces (like
the force from his jetpack pushing him upwards). These forces should be easily
programmable by the game programmer. That is why a physics system is needed in the
cr\^epe game engine. Only very limited/easy physics are needed for Jetpack Joyride,
that is why this is only supported by the cr\^epe game engine.

The protagonist must avoid obstacles. That is why the cr\^epe game engine should also
support a collision system. Again, only very limited/easy collision is needed for
Jetpack Joyride, that is why only very limited/easy collision is supported by the
cr\^epe game engine.

The game must, of course, also be visible to and playable by the user. A rendering
system will take care of rendering (displaying) the game and its GameObjects. An
input system will take care of all the inputs (mouse and keyboard).

Jetpack Joyride also offers audio. A system will take care of the audio in the
cr\^epe game engine.

Particles are very common in Jetpack Joyride, e.g. underneath the jetpack and behind
the rockets. Particles will be supported by the particle system.

The start of a scene is described in a scene. However, the game programmer might also
want to write game logic code which is running during the game (e.g. to switch to a
new scene or to perform a custom action at a collision). For these purposes, Unity
uses scripts. These scripts will also be supported by the cr\^epe game engine.

Finally, as an extra, replay functionality will be supported by the cr\^epe game
engine. A dedicated replay system will be used to support replay.

It turns out that a physics, collision, rendering, input, audio, particle, script,
and replay system are needed to create the Jetpack Joyride game. These systems form
the main part of the \gls{ecs}. The design of these eight systems in combination with
\gls{ecs}, will be briefly discussed in the next parts of this design document.

\section{Design}

% \subsection{Rendering}

% \subsection{Physics}

\subsection{Scripting}

The scripting interface was designed around a `target' \gls{api} (described by
\cref{req:script:interface,req:script:user-class,req:script:direct-instance,req:script:direct-run}).
An example of this \gls{api} is shown below:\noparbreak

\begin{blockcode}
class MyScript : public BehaviorScript {
	void update() {
		// update code here
	}
	// init() also exists, but is empty by default
};

{ // in scene initialization
	GameObject & obj = ...;
	obj.add_component<MyScript>();
}
\end{blockcode}

The above call to \codeinline{GameObject::add_component} cannot work correctly
without significantly increasing the complexity of the component manager, so the
following restrictions were taken into account when creating the script system
architecture:\noparbreak

\begin{itemize}
	\item The first template parameter passed to \codeinline{GameObject::add_component}
		\emph{must} be a base `script \emph{component}' class, so each derived user
		script class is instantiated in the same generic script list.
	\item C++ does not allow passing types (i.e.~\codeinline{MyScript} in this case) as
		function parameters, so a function call like
		\codeinline{add_component<BehaviorScript>(MyScript)} cannot be realized.
\end{itemize}

\subsubsection{Architecture}
\label{sec:scripts:architecture}

The restrictions detailed at the start of this section are mitigated as
follows:\noparbreak

\begin{itemize}
	\item User scripts are split into two classes---
		\begin{enumerate}
			\item a script \emph{interface} class (\codeinline{Script})
			\item a script \emph{component} class (\codeinline{BehaviorScript})
		\end{enumerate}
	\item \codeinline{GameObject::add_component} receives the script \emph{component}
		as template parameter
	\item \codeinline{GameObject::add_component} now always returns a reference to the
		component instance
	\item The script component class has a setter function that takes a template
		parameter for classes derived from the base script \emph{interface} class
\end{itemize}

\Cref{fig:class-scripts} shows the resulting structure as a class diagram. It
contains the following classes:\noparbreak
\begin{description}
	\item[Script] This is the script \emph{interface}, and is used by the game
		programmer to create derived script classes. All virtual methods in this class
		have an empty implementation by default, and are optionally implemented by the
		game programmer.

		This class' virtual methods are protected by default, and a friend relation to
		\codeinline{ScriptSystem} is used to ensure only \codeinline{ScriptSystem} is
		able to call these methods.

		Only classes derived from \codeinline{Script} can be used with
		\codeinline{BehaviorScript::set_script}'s template parameter \codeinline{T}. This
		function returns a reference to the \codeinline{BehaviorScript} instance it was
		called on so it can be chained after the call to
		\codeinline{GameObject::add_component}.

		\codeinline{Script} also has a reference to its parent
		\codeinline{BehaviorScript} instance so components can easily be retrieved using
		the component manager.
	\item[BehaviorScript]
		This is the script \emph{component}, and is given as the template parameter to
		\codeinline{GameObject::add_component}.

		This class also uses a friend relation to \codeinline{ScriptSystem} to restrict
		access to its private reference member \codeinline{script}.
	\item[ScriptSystem] This is the system class that runs the methods implemented in
		the derivative instances of \codeinline{Script}. Described further in
		\cref{sec:scripts:sytem}.
\end{description}

\begin{figure}
	\centering
	\includepumldiag{img/class-scripts.puml}
	\caption{User script class diagram}
	\label{fig:class-scripts}
\end{figure}

\subsubsection{System}
\label{sec:scripts:sytem}

Because most of the complexity in the scripting interface comes from the containers
described in \cref{sec:scripts:architecture}, the script system class itself is
relatively simple. The script system provides a method
\codeinline{ScriptSystem::update} that calls all active script's update functions.

Because of the limitation that types cannot be passed as parameters in C++, the
user-defined script class (derived from \codeinline{Script}) can not directly be
instantiated when adding the component to the component manager. To work around this
limitation, the method \codeinline{BehaviorScript::set_script} was created. This
results in the possibility that an instance of \codeinline{BehaviorScript} does not
reference an instance of \codeinline{Script}. In addition to the non-active script
components, the script system skips over these `invalid' instances. This is
illustrated in \cref{fig:activity-scripts}.

\begin{figure}
	\centering
	\includepumldiag{img/activity-scripts.puml}
	\caption{Script system update method}
	\label{fig:activity-scripts}
\end{figure}

A \gls{poc} for the script system is shown in \cref{poc:scripts}.

\subsection{Audio}

Since writing a custom real-time audio mixing engine is outside the scope of this
project\mref and C++ does not provide a built-in cross-platform audio \gls{api}, the
audio system inside the cr\^epe engine is implemented as a \gls{facade} around an
existing audio library.

\subsubsection{Libraries}
\label{sec:audio:libs}

This subsection compares various standalone audio libraries for suitability. After
searching for libraries (search terms: `dynamic/adaptive audio', `real-time audio',
`audio library', `game audio engine'), several libraries were found. These libraries
were checked against the audio engine requirements \autocite{crepe:requirements} and
then tested by writing the same benchmark-style \gls{poc} using the remaining
qualifying libraries. These \glspl{poc} are detailed in \cref{poc:audio}.

Of these libraries the following were determined to be unsuitable for use in this
project:\noparbreak
\begin{description}
	\item[FMOD \autocite{lib:fmod}] Is proprietary (violates \cref{req:lib:license}).
	\item[PortAudio \autocite{lib:portaudio}] Does not handle mixing.
	\item[miniaudio \autocite{lib:miniaudio}] Tested by implementing a \gls{poc}, but
		dropped due to very limited codec support (WAV, MP3 and FLAC only); Also does not
		have an \gls{api} reference (only programming manual).
	\item[YSE \autocite{lib:yse}] Attempted to write \gls{poc}, but CMake configuration
		in repository is broken; This project seems to have been abandoned.
\end{description}

The only library that remained after these tests is SoLoud \autocite{lib:soloud}. It
is Zlib/LibPng licensed and provides a high-level object-oriented C++ \gls{api}.
\Cref{sec:audio:architecture} describes the \gls{facade} written for this library.

\subsubsection{Architecture}
\label{sec:audio:architecture}

\Cref{fig:class-audio-facade} shows a class diagram of the audio \gls{facade}. It
contains the following classes:
\begin{description}
	\item[SoundContext] This is a wrapper around the \codeinline{SoLoud::soloud}
		`engine' class, and is therefore implemented as a singleton. This ensures the
		audio engine is initialized before \codeinline{Sound} is able to use it.

		This class is friends with \codeinline{Sound}, so only \codeinline{Sound} is able
		to get the \codeinline{SoundContext} instance.
	\item[Sound] This is a wrapper around the \codeinline{SoLoud::Wav} class, and uses
		cr\^epe's \codeinline{Asset} class to load an audio sample instead.
\end{description}

\begin{figure}
	\centering
	\includepumldiag{img/facade-audio.puml}
	\caption{Audio \glsfmtshort{facade} class diagram}
	\label{fig:class-audio-facade}
\end{figure}

A \gls{poc} for the final Audio \gls{facade} is also showcased in \cref{poc:audio}.

\subsection{Save manager}

The save manager \gls{api} is designed to give the game programmer an easy to use
interface for retrieving and storing game-specific data (\cref{req:savemgr}).

Because the engine validation app only stores statistics and highscores, the save
manager is not required to support loading different save files
(\cref{req:savemgr:multi-file}), nor storing complicated data types
(\cref{req:savemgr:types-custom}). The save manager only supports storing simple
types (\cref{req:savemgr:types-scalar,req:savemgr:types-string}).

In order to reduce complexity for the game programmer further, the following
requirements were also set:\noparbreak

\begin{itemize}
	\item Prevent data loss in the case of crashes (\cref{req:savemgr:journalling})
	\item Handle opening/closing/flushing of the underlying file automatically
		(\cref{req:savemgr:file-manage})
	\item Save file variables are uniquely identified (\cref{req:savemgr:var-key})
\end{itemize}

% \subsubsection{Architecture}
% \label{sec:savemgr:architecture}
% 
% \begin{figure}
% 	\centering
% 	\includepumldiag{img/class-savemgr.puml}
% 	\caption{Save manager class diagram}
% 	\label{fig:class-savemgr}
% \end{figure}
% 
% In order to realize \cref{req:savemgr:journalling,req:savemgr:var-key}, a third-party
% key-value database library is used.

\subsection{Global configuration interface}

Because the game programmer only has access to interfaces within the public \gls{api}
namespace (\codeinline{crepe::api}), they would not be able to configure aspects of
engine-internal components. To work around this access restriction, a global
interface was made that stores arbitrary data, which can be accessed both internally
and via the public \gls{api}.

\subsubsection{Architecture}
\label{sec:config:architecture}

The global configuration interface consists of a single singleton class that can be
accessed globally (see \cref{fig:class-config}). This class holds several anonymous
structs, which are used to organize options per system or engine component.

\begin{figure}
	\centering
	\includepumldiag{img/class-config.puml}
	\caption{Global configuration interface class diagram}
	\label{fig:class-config}
\end{figure}

% \subsection{Input}

% \subsection{Physics}

\appendix

\section{\Glsfmtlongpl{poc}}

The full (documented) source code of these \glspl{poc} is available on GitHub
\autocite{crepe:code-repo}.

\subsection{Script system}
\label{poc:scripts}

The script system \gls{poc} \autocite[script example]{crepe:code-repo} consists of
the following:\noparbreak
\begin{itemize}
	\item A user-defined class (\codeinline{MyScript}) derived from
		\codeinline{Script}, which only implements the \codeinline{update()} function.
	\item A main function that---
		\begin{itemize}
			\item Creates a game object with \codeinline{Transform} and
				\codeinline{BehaviorScript} components.
			\item A call to \codeinline{ScriptSystem::update}, which results in
				\codeinline{MyScript::update} being called.
		\end{itemize}
\end{itemize}

Running the \gls{poc} results in the output shown in \cref{fig:poc-output-scripts},
demonstrating that the system works as intended.

\begin{figure}
	\centering
	\fitimg{\includegraphics[scale=0.7]{img/poc-output-scripts.png}}
	\caption{Script system \glsfmtshort{poc} output}
	\label{fig:poc-output-scripts}
\end{figure}

\subsection{Logging utilities}
\label{poc:log}

A small \gls{poc} was written to test the engine's logging functions \autocite[log
example]{crepe:code-repo}. The following calls are used in this example:

\begin{blockcode}
dbg_trace();                                 // the dbg_* macros automatically show
dbg_logf("test printf parameters: %d", 3);   // where the message is coming from
logf(LogLevel::INFO,    "info message");
logf(LogLevel::WARNING, "warning");
logf(LogLevel::ERROR,   "error");
\end{blockcode}

The output of this test is shown in \cref{fig:poc-log}.

\begin{figure}
	\centering
	\includegraphics[scale=0.7]{img/poc-log.png}
	\caption{Logging function outputs}
	\label{fig:poc-log}
\end{figure}

\subsection{Audio}
\label{poc:audio}

A test that consists of the following steps was written for each audio library
mentioned in \cref{sec:audio:libs}:\noparbreak
\begin{enumerate}
	\item Load a background track (Ogg Vorbis)
	\item Load three short samples (WAV)
	\item Start the background track
	\item Play each sample sequentially while pausing and resuming the background track
	\item Play all samples simultaniously
	\item Stop all audio and exit
\end{enumerate}

The repository \autocite{crepe:code-repo} contains two finished \glspl{poc} under the
\codeinline{mwe/audio/} subdirectory for miniaudio and SoLoud. The SoLoud \gls{poc}
was later converted to a full audio \gls{facade}, which is currently part of the
cr\^epe engine. The \gls{poc} using the audio \gls{facade} is available from the same
repository, under the \codeinline{src/example/audio_internal.cpp} file.

\end{document}

